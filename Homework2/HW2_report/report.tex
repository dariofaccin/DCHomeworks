\documentclass[a4paper, 12pt]{report}
\usepackage[T1]{fontenc}
\usepackage[utf8]{inputenc}
\usepackage[english]{babel}
\usepackage{mathtools}
\usepackage{amsfonts}
\usepackage{amsmath}
\usepackage{mathrsfs}
\usepackage{enumitem}
\usepackage{booktabs}
\usepackage{array}
% Avoid paragraph indent
\setlength{\parindent}{0pt}
% Useful floor and ceiling functions
\DeclarePairedDelimiter{\floor}{\lfloor}{\rfloor}
\DeclarePairedDelimiter{\ceil}{\lceil}{\rceil}
% Modified margins
\usepackage[margin=2cm]{geometry}
% This avoids hypenation
\hyphenpenalty=10000
\usepackage{tikz}
\usetikzlibrary{arrows,calc,positioning,shadows,shapes}
\usepackage{graphicx}
\graphicspath{ {figs/} }
\usepackage{subfig}
\captionsetup[figure]{labelfont={bf},name={Figure},labelsep=period}
\captionsetup[table]{labelfont={bf},name={Table},labelsep=period}

\usepackage{float}


\begin{document}
	
\title{Digital Communications and Laboratory \\ Second Homework}
\author{Faccin Dario, Santi Giovanni}
\date{}
\maketitle



\section*{Problem 1}

\section*{Problem 2}
The Doppler spectrum $\mathcal{D}(\lambda)$ represents the power of the Doppler shift for different frequencies $\lambda$.\\
This quantity is defined as the Fourier transform of the autocorrelation function of the impulse response of the channel.\\
The maximum frequency $f_d$ of the Doppler spectrum support is called \textit{Doppler spread} of the channel and it is a measure of the fading rate.
A model which is widely used is the so called \textit{classical Doppler spectrum}:
\begin{equation}
\mathcal{D}(f) =
\begin{cases}
\frac{1}{\pi f_d} \frac{1}{\sqrt{1-(f-f_d)^2}} & |f| \leq f_d \\
0 & \text{otherwise}
\end{cases}
\end{equation}

The Doppler spectrum can be implemented using an IIR filter $h_{ds}$ such that $|\mathcal{H}_{ds}(f)|^2 = \mathcal{D}(f)$.

\begin{thebibliography}{15}
	\bibitem{nevio<3}
	Nevio Benvenuto, Giovanni Cherubini,
	\textit{Algorithms for Communication Systems and their Applications}. 
	Wiley, 2002.
\end{thebibliography}

\end{document}